\documentclass[11pt]{article}
\usepackage[utf8]{inputenc}
\usepackage[T1]{fontenc}
\usepackage[left=1.5cm, right=1.5cm,top=1.5cm, bottom=1.5cm]{geometry}
\usepackage[colorlinks=true, allcolors=blue]{hyperref}
\usepackage{amsmath, amsfonts, lineno}
\linenumbers


\begin{document}
{\bf Model problem}: Let $\hat{\boldsymbol{y}} \in L^2(0, T; \mathbb{R}^N)$ a given target and $\gamma, \nu > 0$ two penalization parameters. We minimize the cost functional
\[\min_{\boldsymbol{y}, \boldsymbol{u}} J(\boldsymbol{y}, \boldsymbol{u}) := \frac12 \int_0^T | \boldsymbol{y}(t) - \hat{\boldsymbol{y}}(t) |^2 \, dt +  \frac{\gamma}2 | \boldsymbol{y}(T) - \hat{\boldsymbol{y}}(T) |^2 + \frac{\nu}{2} \int_0^T |\boldsymbol{u}(t)|^2 \, dt,\]
such that
\[\dot{\boldsymbol{y}}(t) + A \boldsymbol{y}(t) = \boldsymbol{u}(t), 
\quad
\boldsymbol{y}(0) = \boldsymbol{y}_0.\]
Here, $\boldsymbol{y}(t), \boldsymbol{u}(t) \in \mathbb{R}^N$ are column vectors, and $|\cdot|$ is the Euclidean norm. This problem can be seen as the semi-discretization in space of some linear PDE-constrained optimization problem. 


{\bf First-order optimality system}: The standard approach to treat this problem is to introduce a Lagrange multiplier $\boldsymbol{p} \in L^2(0, T; \mathbb{R}^N)$, and write the Lagrangian as
\[\mathcal{L}(\boldsymbol{y}, \boldsymbol{p}, \boldsymbol{u}) = J(\boldsymbol{y}, \boldsymbol{u}) - \int_0^T \boldsymbol{p}^T(t)\Big(\dot{\boldsymbol{y}}(t) + A \boldsymbol{y}(t) - \boldsymbol{u}(t)\Big)\, dt.\]
For any variation $\delta \boldsymbol{y} \in C^{\infty}(0, T; \mathbb{R}^N)$ with $\delta \boldsymbol{y}(0) = 0$, we have
\[\begin{aligned}
\partial_{\boldsymbol{y}}\mathcal{L}[\delta \boldsymbol{y}] = &\int_0^T \big(\boldsymbol{y}(t) - \hat{\boldsymbol{y}}(t)\big)^T\delta \boldsymbol{y}(t) \, dt + \gamma \big(\boldsymbol{y}(T) - \hat{\boldsymbol{y}}(T)\big)^T\delta \boldsymbol{y}(T) \\
&+  \int_0^T \boldsymbol{\dot p}^T(t)\delta \boldsymbol{y}(t) \, dt - \int_0^T (A^T\boldsymbol{p})^T(t) \delta \boldsymbol{y}(t) \, dt  - \boldsymbol{p}^T(T)\delta \boldsymbol{y}(T) + \boldsymbol{p}^T(0)\delta \boldsymbol{y}(0). 
\end{aligned}\]
Equating it to zero gives
\[-\dot{\boldsymbol{p}}(t) + A^T \boldsymbol{p}(t) = \boldsymbol{y}(t) - \hat{\boldsymbol{y}}(t),
\quad
\boldsymbol{p}(T) = \gamma \big(\boldsymbol{y}(T) - \hat{\boldsymbol{y}}(T)\big).\]
Similarly, for any variation $\delta \boldsymbol{u} \in C^{\infty}(0, T; \mathbb{R}^N)$, we have
\[\partial_{\boldsymbol{u}}\mathcal{L}[\delta \boldsymbol{u}] = \int_0^T \nu\boldsymbol{u}(t)^T \delta \boldsymbol{u}(t) \, dt + \int_0^T \boldsymbol{p}^T(t)\delta \boldsymbol{u}(t) \, dt.\]
Equating it to zero gives
\[\boldsymbol{p}(t) + \nu \boldsymbol{u}(t) = 0.\]
We then obtain the optimality system
\[\begin{aligned}
\dot{\boldsymbol{y}}(t) + A \boldsymbol{y}(t) &= \boldsymbol{u}(t), && \boldsymbol{y}(0) = \boldsymbol{y}_0, \\
-\dot{\boldsymbol{p}}(t) + A^T \boldsymbol{p}(t) &= \boldsymbol{y}(t) - \hat{\boldsymbol{y}}(t), &&\boldsymbol{p}(T) = \gamma \big(\boldsymbol{y}(T) - \hat{\boldsymbol{y}}(T)\big), \\
\boldsymbol{p}(t) + \nu \boldsymbol{u}(t) &= 0,
\end{aligned}\]
which can also be written in the integral form as
\[\begin{aligned}
\boldsymbol{y}(t) &= \boldsymbol{y}_0 + \int_0^t \big( -A \boldsymbol{y}(t) + \boldsymbol{u}(t) \big)\, dt, \\
\boldsymbol{p}(t) &= \boldsymbol{p}(T) + \int_t^T \big(-A^T \boldsymbol{p}(t) + \boldsymbol{y}(t) - \hat{\boldsymbol{y}}(t) \big)\, dt, \\
\boldsymbol{u}(t) &= - \boldsymbol{p}(t) / \nu, \quad t\in(0, T).
\end{aligned}\]


{\bf Reduced cost functional}: To enter in the same optimization framework proposed in~\cite{CV2025}, we introduce the solution operator $\mathcal{S} : L^2(0, T; \mathbb{R}^N) \to L^2(0, T; \mathbb{R}^N)$ such that $\mathcal{S} \boldsymbol{u} = \boldsymbol{y}$. Substituting $\mathcal{S} \boldsymbol{u}$ into the cost functional gives 
\[J(\boldsymbol{u}) = J(\mathcal{S} \boldsymbol{u}, \boldsymbol{u}) = \frac12 \int_0^T | \mathcal{S} \boldsymbol{u}(t) - \hat{\boldsymbol{y}}(t) |^2 \, dt +  \frac{\gamma}2 | \mathcal{S} \boldsymbol{u}(T) - \hat{\boldsymbol{y}}(T) |^2 + \frac{\nu}{2} \int_0^T |\boldsymbol{u}(t)|^2 \, dt,\]
where we still use $J$ to denote the reduced cost functional. We can then write the optimization problem as $\min_{\boldsymbol{u}} J(\boldsymbol{u})$ and use the elimination approach proposed in~\cite{CV2025}. 


{\bf Decomposition}: There are two different ways to decompose the problem and write
\[J(\boldsymbol{u}_1, \boldsymbol{u}_2).\]
If we decompose the control variable $\boldsymbol{u}$ into two controls $\boldsymbol{u}_1(t)\in\mathbb{R}^{N_1}$ and $\boldsymbol{u}_2(t)\in\mathbb{R}^{N_2}$ with $N_1 + N_2 = N$. This is exactly in the spirit of~\cite{CV2025}, which corresponds to a space decomposition. In this case, the functional $J: L^2(0, T; \mathbb{R}^{N_1}) \times L^2(0, T; \mathbb{R}^{N_2}) \to \mathbb{R}$. Instead, one can also decompose the time interval $(0, T)$ into two subintervals $Q_1 := (T_0, T_1)$ and $Q_2 := (T_1, T_2)$ with $T_0 = 0$, $T_2 = T$ and $T_1\in(0, T)$. This corresponds to a time decomposition, and the two associated controls $\boldsymbol{u}_1(t)\in\mathbb{R}^{N}$, $t\in Q_1$ and $\boldsymbol{u}_2(t)\in\mathbb{R}^{N}$, $t\in Q_2$. In this case, the functional $J: L^2(T_0, T_1; \mathbb{R}^{N}) \times L^2(T_1, T_2; \mathbb{R}^{N}) \to \mathbb{R}$.
%LD: sorry for complexify the notation, may be convenient in case of having many subintervals.


As we are interested in the second case, let us derive the algorithm following the approach proposed in~\cite{CV2025}. Assuming that for every $\boldsymbol{u}_1$ the equation $\nabla_{\boldsymbol{u}_2}J(\boldsymbol{u}_1, \boldsymbol{u}_2) = 0$ admits a unique solution $\boldsymbol{u}_2$ and that  $\nabla_{\boldsymbol{u}_2\boldsymbol{u}_2}J(\boldsymbol{u}_1, \boldsymbol{u}_2)$ is invertible for every $(\boldsymbol{u}_1, \boldsymbol{u}_2)$, then applying the implicit function theorem, there exists a continuously differentiable mapping $h: L^2(T_0, T_1; \mathbb{R}^{N}) \to L^2(T_1, T_2; \mathbb{R}^{N})$ such that we can eliminate $\boldsymbol{u}_2$ and obtain $\nabla_{\boldsymbol{u}_2}J(\boldsymbol{u}_1, h(\boldsymbol{u}_1)) = 0$. We may apply the Newton iteration to solve the reduced optimality condition $F(\boldsymbol{u}_1) = \nabla_{\boldsymbol{u}_1}J(\boldsymbol{u}_1, h(\boldsymbol{u}_1)) = 0$. For iteration index $k = 0, 1, \ldots$, one solves
\[\boldsymbol{u}_1^{k+1} = \boldsymbol{u}_1^{k} - \big(JF(\boldsymbol{u}_1^{k})\big)^{-1}\nabla_{\boldsymbol{u}_1}J(\boldsymbol{u}_1^k, h(\boldsymbol{u}_1^k)),\]
which is exactly what has been shown in~\cite[Eq. 4]{CV2025}. 
%LD: if I understand correctly, this is not exactly what we are aiming for. 


As also discussed in~\cite{CV2025}, our ultimate goal is to solve the optimization problem. Thus, one can also perform such variable elimination on the objective function $J$, that is, $\tilde J(\boldsymbol{u}_1) := J(\boldsymbol{u}_1, h(\boldsymbol{u}_1))$, and then apply a gradient descent method to solve the minimization problem as  
\[\boldsymbol{u}_1^{k+1} = \boldsymbol{u}_1^{k} - \alpha \nabla \tilde J(\boldsymbol{u}_1^{k}).\]
with $\alpha$ the step size satisfies $\tilde J(\boldsymbol{u}_1^k - \alpha \nabla \tilde J(\boldsymbol{u}_1^{k})) < \tilde J(\boldsymbol{u}_1^k)$. 


{\bf Algorithm}: The solving process can be resumed as:
\begin{enumerate}
\item For a given $\boldsymbol{u}_1^{k}$, one can find the state variable $\boldsymbol{y}_1^{k}$ with
\[\boldsymbol{y}_1^k(t) = \boldsymbol{y}_0 + \int_0^t \big( -A \boldsymbol{y}_1^k(t) + \boldsymbol{u}_1^k(t) \big)\, dt.\]
This consists in applying the solution operator $\mathcal{S}_1 \boldsymbol{u}_1^{k}$.

\item Using the fact that $\boldsymbol{y}_2^{k}(T_1) = \boldsymbol{y}_1^{k}(T_1)$, one solves the system
\[\begin{aligned}
\boldsymbol{y}_2^{k}(t) &= \boldsymbol{y}_2^{k}(T_1) + \int_{T_1}^{t} \big( -A \boldsymbol{y}_2^{k}(t) + \boldsymbol{u}_2^{k}(t) \big)\, dt, \\
\boldsymbol{p}_2^{k}(t) &= \boldsymbol{p}_2^{k}(T_2) + \int_t^{T_2} \big(-A^T \boldsymbol{p}_2^{k}(t) + \boldsymbol{y}_2^{k}(t) - \hat{\boldsymbol{y}}(t) \big)\, dt, \\
\boldsymbol{u}_2^{k}(t) &= - \boldsymbol{p}_2^{k}(t) / \nu, \quad t\in(T_1, T_2).
\end{aligned}\]
The above part consists in evaluating the mapping $\boldsymbol{u}_2^k = h(\boldsymbol{u}_1^{k})$. 

\item Update the control variable in $Q_1$ with
\[\boldsymbol{u}_1^{k+1} = \boldsymbol{u}_1^{k} - \alpha \Big( \nabla_{\boldsymbol{u}_1} \tilde J\big(\boldsymbol{u}_1^{k}, h(\boldsymbol{u}_1^{k})\big) +  \nabla_{\boldsymbol{u}_2} \tilde J\big(\boldsymbol{u}_1^{k}, h(\boldsymbol{u}_1^{k})\big) h'(\boldsymbol{u}_1^{k})\Big).\]
As the implicit mapping $h$ fulfils $\nabla_{\boldsymbol{u}_2}J(\boldsymbol{u}_1^{k}, h(\boldsymbol{u}_1^{k})) = 0$. The update can then be written as
\[\boldsymbol{u}_1^{k+1} = \boldsymbol{u}_1^{k} - \alpha  \nabla_{\boldsymbol{u}_1} \tilde J\big(\boldsymbol{u}_1^{k}, h(\boldsymbol{u}_1^{k})\big) = \boldsymbol{u}_1^{k} - \alpha (\boldsymbol{p}_1^k + \nu \boldsymbol{u}_1^k ),\]
where $\boldsymbol{p}_1^k$ is given by
\[\boldsymbol{p}_1^k(t) = \boldsymbol{p}_1^k(T_1) + \int_t^{T_1} \big(-A^T \boldsymbol{p}_1^k(t) + \boldsymbol{y}_1^k(t) - \hat{\boldsymbol{y}}(t) \big)\, dt,\]
with $\boldsymbol{p}_1^k(T_1) = \boldsymbol{p}_2^k(T_1)$.
\end{enumerate}
Here, we are in the case with an exact solve of $\nabla_{\boldsymbol{u}_2}J(\boldsymbol{u}_1, \boldsymbol{u}_2) = 0$, as $\nabla_{\boldsymbol{u}_2}J(\boldsymbol{u}_1, \boldsymbol{u}_2) = \nu \boldsymbol{u}_2^{k} +\boldsymbol{p}_2^{k}$. 

\bibliography{biblio}
\bibliographystyle{siam}
\end{document}